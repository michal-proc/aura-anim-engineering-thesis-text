\section{Introduction}

\subsection{Purpose of the chapter}
The purpose of this chapter is to present a clear description of the scope of the project. We include the requirements that define the functionality and performance of the text-to-video generation system
and analyze the feasibility of the solutions in the context of computational complexity of the problem.
The chapter outlines what the system is expected to do, how it should behave under different conditions, and the quality standards it should meet. These requirements provide a
a foundation for the design of the system, its implementation, and evaluation. We specify both functional and non-functional requirements to provide a clear set of
expectations for the users of the application.


\section{Project scope}

This section defines the boundaries of the system implemented in this thesis. Here, we aim to specify the functionalities meant to be
included in the prototype and clarify the limitations under which it will operate. We also describe the possible extensions of the
system that are outside of the scope of this thesis.

\subsection{Planned functionality}

The text-to-video generation system is meant to be a software application capable of automatically producing short videos from user-provided text prompts.
The system should be designed to interpret natural language input, extract key elements such as subjects, actions, and contexts, and generate a video that visually represents the described scene.


Within the scope of the project we intend to develop a web application which will be separated into a SPA web interface and
backend servers providing a scalable way to generate videos. To achieve this we plan to implement several groups of functionalities:
\begin{itemize}
    \item \emph{User registration, authentication, and account management}: These are crucial to distinguish between individual users of the application and provide personalized experience.
    \item \emph{Submission of text prompts and generation parameters}: Users need a clear and friendly way to submit their text prompts and customize basic generation parameters to suit their needs.
    \item \emph{Execution of video generation and storage of the results}: A text prompt along with a set of parameters needs to be transformed into a video and saved in the storage.
    \item \emph{Gallery and management of generated videos}: Users need an intuitive way to view their videos and download them.
    \item \emph{Web interface integrating backend functionalities}: Implements a user interface for the application delegating the video generation and the rest of the functionalities to the application backend.
\end{itemize}

The precise description of these planned functionalities is included in the Section~\ref{sec:functional-requirements}.

\subsection{Excluded functionality}
 
Certain features and capabilities have been intentionally excluded from the scope of the project to make its development manageable
given its time constraints, number of project participants, and hardware limitations. These exclusions ensure that the focus of the project
is on demonstrating the feasibility of video generation in practical software implementations rather than on building a system that includes
the maximum amount of features possible.
 
We have identified the following functionality that is available in commercial products providing text-to-video generation, but that we do not
intend to include in the scope of our project:
\begin{itemize}
    \item \emph{Uploading of users' own videos and their management}: The storage and gallery functionality is meant to only respectively store and display the videos that were generated by the application. We do not intend to accept videos uploaded by the users of the application.
    \item \emph{Organization of videos into folders}: Our intention is to show the user a timeline of their generated videos and we do not want to complicate the application by reimplementing functionality similar to personal cloud storage functionality.
    \item \emph{Generation of images}: The research problem that we are trying to tackle in this thesis is designing an application for solving the text-to-video generative AI problem. Some commercial text-to-video generation applications bundle a text-to-image generation functionality to be more versatile. We do not wish to pursue this path since it is not the goal of our thesis.
    \item \emph{Audio generation for videos}: State-of-the-art text-to-video models like Veo 3 or Sora 2 are able to add audio relevant to the videos they generate. This is still very much a research frontier and we have neither the expertise nor enough resources to pursue it.
\end{itemize}
 
\subsection{Optional functionality}
 
While certain features have been excluded from the initial scope, the modular architecture of the system is designed to accommodate future enhancements if additional development time becomes available. These potential extensions would add value to the application without compromising its core functionality:
 
\begin{itemize}
    \item \emph{Light social networking features}: A community gallery where users can optionally publish their generated videos for others to view and rate. This would include basic privacy controls allowing users to choose which videos to share publicly and features for browsing and discovering content created by other users.
    \item \emph{Prompt templates and suggestions}: A library of pre-designed prompt templates categorized by themes such as nature, abstract art, or specific visual styles. This would help users who are unfamiliar with prompt engineering to achieve better generation results.
    \item \emph{Video editing capabilities}: Basic post-generation editing tools allowing users to trim, crop, or apply simple filters to generated videos without leaving the application.
    \item \emph{Batch generation}: The ability to queue multiple prompts for sequential generation, enabling users to explore variations of a concept or generate multiple videos overnight.
    \item \emph{Generation history and analytics}: Detailed statistics about generation parameters and their outcomes, helping users understand which configurations produce their preferred results.
    \item \emph{Collaborative features}: Functionality allowing multiple users to work together on video generation projects, sharing prompts and parameter configurations within teams.
\end{itemize}
 
These extensions represent natural evolution paths for the system that align with user needs while maintaining the focus on the core capability.

\subsection{Performance limitations}
In practical aspects of deep learning one of the central issues is the vast computational resources required to get the model weights (training) and use the model (inference).
The generative problem that we are tackling is one of the most demanding since it not only requires for individual frames to be generated (as is done in text-to-image models)
but the model also needs to handle temporal dimension. Because of the extra dimension, generating a video requires higher memory bandwidth, more VRAM for storing activations, more storage
for saving weights/intermediate data, and possibly multiple GPUs to distribute the workload and get the result in acceptable time. Memory requirements for generating videos using open source,
end-to-end architectures currently push the limits of VRAM capacity of consumer GPUs at reasonable prices. This is significantly more severe for GPUs that are not from the RTX xx90 series. Our
development hardware includes one GPU with 12 GB of VRAM and another with 8 GB. Accordingly our initial options for testing text-to-video models are limited and we decided to focus our attention
on pipeline-based text-to-video models that repurpose text-to-image diffusion models for video generation. If we get access to higher-end GPUs that include much more computing
power and VRAM we intend to experiment with open source end-to-end models and compare their outputs to our pipeline-based approach.

\section{System users and external systems}

In this section we will describe different types of users who will interact with the system and other systems
that it will depend on.

\subsection{System users}
Our system aims to provide video generation capabilities through an easy-to-use and elegant web-based user interface. It means to be
a self-contained system that does not require any administration by providing users with control of the videos they generated and giving
them full control over any modifications or deletions of data stored in their accounts, including the account itself.

Accordingly, the only type of user that we have identified for the system is the \emph{end user}, who interacts directly with the
application through a web interface. After registration and authentication, the end user gains complete access to the functionality
offered by the system. This includes submitting text prompts for video generation, adjusting generation parameters, downloading the generated
video, and viewing the results of previous generations.

The planned design assumes that users have no specialized knowledge of AI models or video processing tools, therefore
emphasis will be placed on usability and accessibility of the application for non-technical users.

\subsection{External systems}
The planned system will operate as a self-contained application with all processing tasks and storage
managed within the system infrastructure. It will not depend on external systems for user interaction, authentication,
or video generation.

The only interactions that will occur with external systems will involve downloading external resources such as weights used for various models
used in the video generation process. Specifically, the system will retrieve pre-trained model weights from repositories such as
\emph{Hugging Face} and \emph{GitHub}. These requests will be limited to downloading publicly accessible assets using secure HTTP requests.
No user data or content generated by the system will be transmitted to these external sources.

Apart from these one-directional data transfers the system will not depend on any third-party APIs or external computational services.
In particular, the storage of user data and generated videos will be handled on the servers which host the application.


\section{Functional requirements}
\label{sec:functional-requirements}

This section specifies the functional requirements of the system. Functional requirements define the expected behavior of the system and describe
the services it provides to its users. The requirements are organized around the main functionalities of the application such as user account
management, video generation workflow, job management, distributed execution, storage and retrieval of user data and generated content.
For each functional area the following subsections describe the specific features and their purpose.

\subsection{Account management}
\begin{itemize}
    \item[FR-1] The system shall allow users to create a new account by providing a unique email address and username, as well as a password.
    \item[FR-2] The system shall prevent the creation of new accounts that use an email address or username already associated with an existing account.
    \item[FR-3] The system shall allow users to recover access to their account by sending a password reset link to the email address registered with the account.
    \item[FR-4] The system shall allow users to log in using their registered credentials.
    \item[FR-5] The system shall allow users to temporarily deactivate their account without deleting its associated data from the system.
    \item[FR-6] The system shall allow users to reactivate their account and return the account status to the state from before the deactivation.
    \item[FR-7] The system shall allow users to permanently delete their account, removing all related data from the system.
\end{itemize}

These requirements ensure that users can securely manage the entire lifecycle of their accounts, from registration and authentication
to temporary deactivation and permanent deletion.

\subsection{Video generation workflow}
\begin{itemize}
    \item[FR-8] The system shall allow users to submit a text prompt describing the content of the desired video.
    \item[FR-9] The system shall allow users to configure generation parameters such as video length, resolution, aspect ratio, and frame rate.
    \item[FR-10] The system shall validate the submitted prompt and parameters before starting video generation.
    \item[FR-11] The system shall initiate video generation process based on the validated parameters provided by the user.
    \item[FR-12] The system shall track the progress of the video generation process and update its status in the database as the job progresses.
    \item[FR-13] The system shall handle failures in the video generation process by reporting errors and preventing corrupted or incomplete outputs from being stored.
    \item[FR-14] The system shall notify the user when video generation is complete and the resulting video is available.
    \item[FR-15] The system shall store the generated video and its metadata for later retrieval.
\end{itemize}

These requirements define the complete workflow for transforming a user-provided text prompt into a video. They ensure
that the system can validate and process user input, perform the generation process reliably, and provide the user with access
to the result once it is available.

\subsection{Gallery and retrieval}
\begin{itemize}
    \item[FR-16] The system shall provide users with a gallery view of the videos they generated.
    \item[FR-17] The system shall present the gallery of generated videos showing them in chronological order.
    \item[FR-18] The system shall allow users to view the prompt that generated the video.
    \item[FR-19] The system shall allow users to preview and play the generated videos within the application.
    \item[FR-20] The system shall allow users to download generated videos.
    \item[FR-21] The system shall allow users to share their generated videos on social media platforms. 
    \item[FR-22] The system shall allow users to delete generated videos.
\end{itemize}

These requirements ensure that users can access and manage their generated videos as well as view the text prompt that
was used to create them.

\subsection{Video generation job management}
\begin{itemize}
    \item[FR-23] The system shall monitor the execution status of each video generation job.
    \item[FR-24] The system shall update the stored status of a video generation job when its state changes.
    \item[FR-25] The system shall notify the user when a video generation job succeeds or fails.
    \item[FR-26] The system shall prevent incomplete or failed jobs from producing user visible outputs except for informing the user that the video generation failed.
\end{itemize}

These requirements ensure that the system can reliably monitor the video generation process, inform
the user about the generation status, and maintain consistency between the video generation job status and videos
displayed to the user.

\subsection{User interface capabilities}
\begin{itemize}
    \item[FR-27] The system shall provide a web-based graphical user interface that allows users to access all application functions after authentication.
    \item[FR-28] The system shall organize the main functions of the application into clearly identifiable sections, including areas for prompt submission, video generation progress, and the gallery of generated videos.
    \item[FR-29] The system shall display the current state of user actions, including progress indicators for video generation, and status notifications when the generation is completed.
    \item[FR-30] The system shall present error messages in a clear manner, indicating when user input is invalid or when generation fails.
    \item[FR-31] The system shall provide users with consistent navigation elements that allow users to move between the major sections of the application.    
\end{itemize}

These requirements specify how the system presents and communicates its functionality to users. They ensure that the application
interface is clear, informative, and intuitive to use.

\section{Non-functional requirements}

This section specifies the non-functional requirements of the system.
Non-functional requirements define the quality attributes and constraints that govern how the system performs its functions, rather than what it does.

\subsection{Scalability requirements}
\begin{itemize}
    \item[NFR-1] The system architecture shall support horizontal and vertical scaling to accommodate increasing numbers of users and video generation requests.
    \item[NFR-2] The system shall distribute video generation tasks across multiple GPU workers (if available) to optimize resource utilization.
\end{itemize}

These requirements ensure that the system can grow to meet increasing demand by adding computational resources and distributing workload effectively across available infrastructure.

\subsection{Reliability requirements}
\begin{itemize}
    \item[NFR-3] The system shall implement health checks with automatic restart of failed components within 60 seconds.
    \item[NFR-4] The system shall preserve generation tasks state in case of unexpected shutdowns or restarts.
\end{itemize}

These requirements ensure that the system remains available and operational, can recover gracefully from failures, and maintains data consistency even during unexpected disruptions.

\subsection{Usability requirements}
\begin{itemize}
    \item[NFR-5] The system shall provide an intuitive interface requiring no more than 5 clicks to initiate video generation.
    \item[NFR-6] Error messages shall be displayed in clear, non-technical language with actionable guidance.
    \item[NFR-7] The system shall provide visual progress indicators during video generation processes.
\end{itemize}

These requirements ensure that the system is easy to use and communicates issues in an understandable manner.

\subsection{Maintainability requirements}
\begin{itemize}
    \item[NFR-8] The system shall use modular architecture allowing individual components to be updated independently without affecting other parts of the system.
    \item[NFR-9] The system shall provide structured logging with severity levels for troubleshooting and monitoring.
    \item[NFR-10] The system shall be deployable on Docker containers.
    \item[NFR-11] The system shall maintain comprehensive API documentation for all endpoints.
\end{itemize}

These requirements ensure that the system can be easily updated, debugged, and maintained over time without disrupting service availability.

\subsection{Resource constraints}
\begin{itemize}
    \item[NFR-12] Individual video generation tasks shall operate within 12GB VRAM limits for GPU processing.
    \item[NFR-13] Individual video generation tasks shall operate within 16GB RAM limits for CPU processing.
\end{itemize}

These requirements ensure that the system operates within defined hardware and storage limitations, preventing resource exhaustion and maintaining system stability.


\section{Usage and testing scenarios}

This section presents the main user interaction scenarios that describe how the system is expected to be used in real conditions.
Each scenario defines the sequence of steps performed by the user and the system's corresponding behavior, forming a foundation for acceptance and integration testing.
They also define the basis for the testing phase, as each scenario corresponds to one or more test cases verifying its successful execution.

The structure of these scenarios follows the \textit{use case template} defined in the UML (Unified Modeling Language) methodology and compliant with the IEEE 830 and ISO/IEC/IEEE 29148 standards for software requirement documentation. 
This approach provides a clear link between system functionality, user interactions and validation procedures.

Every scenario includes:
\begin{itemize}
    \item \textbf{Preconditions} - assumptions that must hold before the scenario begins,
    \item \textbf{Scenario} - step-by-step interaction between the user and the system,
    \item \textbf{Postconditions} - the resulting state of the system or user data,
    \item \textbf{Testing goal} - the expected outcome used as a reference during validation.
\end{itemize}

\subsection{Generating videos}

\textbf{U-1: User views generation parameters}

\textbf{Preconditions:} The user has opened the video generation window.

\textbf{Scenario:}
\begin{enumerate}
    \item The user opens the video generation interface.
    \item The system displays all available parameters for the generation process.
    \item The user can hover over or select any parameter to view an explanatory tooltip or text description.
    \item Each parameter is described in a clear and understandable way, allowing users unfamiliar with the generation process to configure them correctly.
\end{enumerate}

\textbf{Postconditions:}
Parameters and their descriptions are visible and accessible to the user within the interface.

\textbf{Testing goal:}
Verify that all generation parameters have visible, understandable descriptions and users can access this information easily before starting the generation process.

\medskip

\textbf{U-2: The user generates and manages a new video}

\textbf{Preconditions:} The user is logged into the system.

\textbf{Scenario:}
\begin{enumerate}
    \item The user navigates to the ``Create Video`` page.
    \item The user enters a text prompt and other generation parameters.
    \item The system validates the input and starts the generation process.
    \item A progress indicator is displayed, showing each stage of generation.
    \item Once the process is completed, the system notifies the user and displays the generated video in a preview window.
    \item The user can view or download the generated video.
    \item The user can share the video on social media platforms.
    \item The generated video becomes available in the user's personal gallery and remains visible after re-logging into the system.
\end{enumerate}

\textbf{Postconditions:}
The generated video is stored in the user's account and remains available for future access, management and sharing.

\textbf{Testing goal:}
Verify that a logged-in user can generate, preview, download and share a video on social media platforms and that the video persists in their personal gallery.

\medskip

\textbf{U-3: User generates multiple videos simultaneously}

\textbf{Preconditions:} The user is logged in and has access to the video generation interface and can initiate multiple generation tasks.

\textbf{Scenario:}
\begin{enumerate}
    \item The user starts generating a video using a selected prompt and parameters.
    \item While the first generation is in progress, the system displays a badge or indicator showing the active generation status.
    \item The user initiates one or more additional video generations in parallel.
    \item The system queues or processes multiple generation tasks concurrently without reducing performance or responsiveness of the interface.
    \item Once the processes complete, each generated video becomes available for preview or download.
\end{enumerate}

\textbf{Postconditions:}
Multiple videos are successfully generated in parallel without affecting the application's responsiveness or causing backend issues.

\textbf{Testing goal:}
Verify that users can start and manage multiple video generations at once and that the system handles concurrent requests without performance degradation or errors.

\medskip

\textbf{U-4: System handles concurrent generation from multiple users}

\textbf{Preconditions:} Multiple users are simultaneously generating videos through the application interface.

\textbf{Scenario:}
\begin{enumerate}
    \item Several users initiate independent video generation processes at the same time.
    \item The backend architecture distributes tasks efficiently among available resources.
    \item The system maintains stable performance and ensures isolation between user sessions.
    \item Each user receives progress feedback for their individual generation tasks.
    \item All requested videos are generated successfully without failures or slowdowns.
\end{enumerate}

\textbf{Postconditions:}
The system remains stable and responsive under concurrent load from multiple users.

\textbf{Testing goal:}
Verify that the application's architecture can handle simultaneous generation requests from many users without crashes, bottlenecks, or performance loss.

\subsection{Managing videos}

\subsubsection{U-5: User shares a video}

\textbf{Preconditions:} The user is logged in and has at least one generated video in their gallery.
\textbf{Scenario:}
\begin{enumerate}
    \item The user opens their gallery and selects a generated video.
    \item The system displays sharing options available for the selected video.
    \item The user selects a desired social media platform for sharing.
    \item The system initiates the sharing process through the platform's integration interface.
    \item The system publishes the video on the selected social media platform.
\end{enumerate}
\textbf{Postconditions:}
The selected video is successfully published on the chosen social media platform and visible on the user's profile.
\textbf{Testing goal:}
Verify that a logged-in user can share a generated video directly from the application to supported social media platforms and that the video appears correctly on the selected profile.

\subsubsection{U-6: User deletes a video}

\textbf{Preconditions:} The user is logged in and has at least one video available in their gallery.

\textbf{Scenario:}
\begin{enumerate}
    \item The user navigates to their video gallery.
    \item The system displays all existing videos.
    \item The user selects a video to remove.
    \item The system asks for confirmation before deletion.
    \item After confirmation, the selected item is permanently deleted.
    \item The gallery view refreshes and no longer displays the deleted item.
\end{enumerate}

\textbf{Postconditions:}
The deleted video is permanently removed from the user's account and is no longer visible or accessible in any part of the application.

\textbf{Testing goal:}
Verify that users can permanently delete videos from their collection and that the removed items are no longer visible or recoverable within the application.

\subsection{Account management}

\subsubsection{U-7: User registers a new account}

\textbf{Preconditions:} The user is not logged in and has access to the registration form.

\textbf{Scenario:}
\begin{enumerate}
    \item The user navigates to the registration page.
    \item The user provides a valid e-mail address and password, then submits the form.
    \item The system sends a verification e-mail containing an activation link.
    \item The user opens the received e-mail and clicks the activation link.
    \item The system verifies the link and activates the new account.
    \item The user is notified that registration has been successfully completed and can now log in.
\end{enumerate}

\textbf{Postconditions:}
A verified account is created in the system, allowing the user to log in and access all authenticated functionalities.

\textbf{Testing goal:}
Verify that a new user can register with valid credentials, receive a verification e-mail and activate the account successfully.

\subsubsection{U-8: User logs into the system}

\textbf{Preconditions:} The user has a verified account.

\textbf{Scenario:}
\begin{enumerate}
    \item The user navigates to the login page.
    \item The user enters valid login credentials (e-mail and password).
    \item The user optionally selects the ``Remember me`` option to stay logged in across sessions.
    \item The system validates the credentials and grants access to the application.
    \item The user is redirected to the main interface with access to their personal gallery and profile options.
\end{enumerate}

\textbf{Postconditions:}
The user is authenticated and has access to all functionalities available to logged-in users.

\textbf{Testing goal:}
Verify that users can log in successfully using valid credentials and remain logged in if the ``Remember me`` option is selected.

\subsubsection{U-9: User resets the account password}

\textbf{Preconditions:} The user has a registered account and access to their e-mail address.

\textbf{Scenario:}
\begin{enumerate}
    \item The user opens the password reset page and enters their e-mail address.
    \item The system sends a password reset e-mail containing a secure link.
    \item The user opens the e-mail and follows the link to the password reset form.
    \item The user provides a new password and submits the form.
    \item The system updates the credentials and confirms that the password has been changed successfully.
\end{enumerate}

\textbf{Postconditions:}
The user's password is updated and they can log in using the new credentials.

\textbf{Testing goal:}
Verify that users can reset their password through the e-mail-based recovery process and regain access to their account.

\subsubsection{U-10: User permanently deletes account with all data}

\textbf{Preconditions:} The user is logged in and has an active account containing videos and other personal data.

\textbf{Scenario:}
\begin{enumerate}
    \item The user navigates to the account settings section.
    \item The user selects the option to permanently delete the account.
    \item The system displays a confirmation prompt informing that all data, including videos and settings, will be irreversibly removed.
    \item The user confirms the action.
    \item The system deletes the user account and all related data from the database.
    \item The session is terminated and the user is redirected to the homepage with a message confirming successful deletion.
\end{enumerate}

\textbf{Postconditions:}
The user account and all associated data are permanently removed from the system. The user can no longer log in with the deleted credentials.

\textbf{Testing goal:}
Verify that the system completely removes the user's account and all dependent data and that access is fully revoked after deletion.

\medskip

\subsubsection{U-11: User deactivates and reactivates account}

\textbf{Preconditions:} The user is logged in and has an active account.

\textbf{Scenario:}
\begin{enumerate}
    \item The user navigates to the account settings section.
    \item The user selects the option to temporarily deactivate their account.
    \item The system displays information about the consequences of deactivation, such as hidden videos and disabled sharing.
    \item The user confirms deactivation.
    \item The system updates the account status to ``deactivated`` and logs the user out.
    \item The user later returns to the login page and provides their credentials.
    \item The system detects the deactivated status and displays an option to reactivate the account.
    \item The user confirms reactivation.
    \item The system restores the account and all related data to the state before deactivation.
\end{enumerate}

\textbf{Postconditions:}
The account is successfully deactivated and later reactivated with all user data preserved.

\textbf{Testing goal:}
Verify that the system correctly handles account deactivation and reactivation, preserving user data and restoring access after reactivation.

The presented testing scenarios cover all essential functionalities of the system, ensuring that each user interaction path can be verified.
They provide a structured basis for assessing the correctness of system behavior and allow for effective evaluation of its overall performance and responsiveness.

\FloatBarrier
\begin{figure}[h!]
    \centering
    \includegraphics[width=0.7\textwidth]{figures/test-requirements}
    \caption{Mapping between user test scenarios and functional requirements}
    \label{fig:image-test-requirements}
\end{figure}
\FloatBarrier

\section{Summary}

This chapter provided a comprehensive specification of the system requirements and usage scenarios for the text-to-video generation application.
It established the foundation for the subsequent design, implementation and testing phases by defining both the functional and non-functional aspects of the system.
The presented requirements describe what the system should do, how it should perform and under what constraints it will operate.
The usage scenarios illustrated typical user interactions, forming a basis for acceptance testing and validation of the system's behavior in real-world conditions.